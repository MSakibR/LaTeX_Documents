\noindent{\Large \textbf{4. Conclusion}}\\[0.5em]
This project managed to create something using renewable energy
smart watering setup that uses two-way sun tracking,
sensor automation plus internet-connected tracking - using
a tiny ESP32 chip, light-sensing solar follow mode but
using water levels to manage watering, the setup showed
clear gains in power use along with water
keeping things safe. Tests showed both directions work
monitoring rising output from sunlight energy, whereas self-operating
Watering got cut down so no extra waste happened.

\vspace{0.5em}

\noindent\textbf{Technical Achievements:}
\begin{itemize}
    \item Setting up a solar tracker that moves on two directions
    with light detectors plus small spinning machines.
    \item Automated irrigation control based on real-time soil mois-
    ture measurement.
    \item IoT-based dashboard enabling remote monitoring of en-
    vironmental parameters.
    \item Steady power away from the grid using sunlight plus a battery
    tery power management.
\end{itemize}

\vspace{0.5em}

\noindent\textbf{Social Relevance:}\\
The new setup works well for countryside farms,
folks, people growing food on roofs, also neighborhoods where water’s hard to come by or
in remote areas. Cutting down on physical work while making better use of water
use, while working apart from regular power
sources - this effort supports farming that lasts, using methods kinder on the planet
uses simple methods that help small farms go digital without high costs.

\vspace{0.5em}

\noindent\textbf{Future Scope:}
\begin{itemize}
    \item Linking smart weather forecasts with AI tools
    irrigation scheduling.
    \item Setting up automatic fertilizer delivery using irrigation systems,
    enhanced crop management.
    \item Adding GSM or SMS alerts where there's no signal, using backup networks when needed - keeps updates running even if main lines fail
    internet connectivity.
    \item Building a special phone app with
    Firebase or Flutter - to boost access.
\end{itemize}
