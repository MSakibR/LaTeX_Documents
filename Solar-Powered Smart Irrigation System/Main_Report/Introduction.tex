% ---------------- I. Introduction ----------------
\noindent{\Large \textbf{1.Introduction}}\\[0.5em]

\noindent{\large \textbf{1.1 Background \& Motivation}}\\[0.3em]
Farming today’s hit by less water, pricier energy, yet folks still rely on old-school watering methods that waste a lot. Outdated setups guzzle way too much H2O while leaning hard on electric grids or fuel-powered pumps - neither cheap nor green. Meanwhile, standard solar panels just sit there; they don’t follow the sun, so they miss out on peak light, slashing how much juice they make.

To tackle these issues, the project introduces a solar-powered smart watering setup combining two-way sun tracking with auto irrigation based on soil dampness. Using clean energy alongside sensors, it’s built to boost power efficiency while cutting down excess water use. The goal is to deliver an eco-friendly tech option suited for small farms or countryside farming needs.
\vspace{0.8em}

\noindent{\large \textbf{1.2 Problem Statement}}\\[-0.5em]
\begin{itemize}
    \item Frozen solar setups can't follow the sun's path - instead, they stay put while light moves across the sky
    \item Overall, this leads to less energy being produced.
    \item Hand-watering often causes uneven moisture levels plus waste of supplies
    \item waste - also possible harm to crops.
    \item Most countryside spots don't have steady power, so automatic
    \item Farming setups paired together are tough to set up.
\end{itemize}

\vspace{0.8em}

\noindent{\large \textbf{1.3 Objectives}}\\[0.3em]
The main goals here include these points:
\begin{itemize}
    \item Build a two-way solar tracker so it follows the sun better
    \item solar panels grab sunlight to make power.
    \item Automate watering by checking soil dampness live
    \item surements.
    \item Run the whole setup on sunlight, storing extra juice in a battery
    \item keep a spare ready so things run smooth.
    \item Incorporate IoT-based monitoring of environmental pa-
    \item like dampness in the ground, how warm it is, or how thick the air feels,
    \item rain yet gas levels.
\end{itemize}

\vspace{0.8em}

\noindent{\large \textbf{1.4 Scope}}\\[0.3em]
\begin{itemize}
    \item Good for tiny farms or growing stuff on rooftops,
    \item and home gardening.
    \item Built to work in remote areas without power or money
    \item where there's not much power available.
    \item Can grow later on, maybe adding smart watering using AI tools down the line
    \item tuning setups, forecast tools with clever automation
    \item farming systems.
\end{itemize}

\vspace{0.8em}


\noindent{\large \textbf{1.5 Related Works}}\\[0.3em]
Recent research points to fast progress in farming tech using sensors and internet-connected watering systems. Obaideen’s team found these smart setups cut down water waste by tracking ground dampness, climate conditions, or crop needs on the fly - helping meet global clean water goals. They stressed how radio-linked sensors plus automated decisions shape smarter irrigation.

García’s team showed moisture sensors using capacitance, combined with internet-linked tiny computers, could cut farm water waste by about a third. These results pointed out how crucial solid signal links are, along with devices that use very little power.

Pawar et al. \cite{IRJET2018} built an affordable irrigation setup using a small control unit; their tests showed automatic watering based on soil dampness cuts down hands-on effort while boosting harvest uniformity - so this kind of tech fits well in countryside farming.

Some hands-on examples of two-way solar trackers are already out there on the web. Instead of just one direction, these setups adjust panels both horizontally and vertically. One project from Electronics Workshops \cite{electronicsworkshops21} used light sensors linked to servos to boost panel output. On another note, a video demo \cite{youtubeSmartIrrigation21} tied this kind of movement to soil dampness readings for smart watering.

Together, these studies prove that using smart devices, real-time data from sensors, or sun-driven monitoring really boosts how well watering systems work while making better use of clean power - backing up the combined method in our suggested setup.
